% $date: 2016-07-03

\section*{Площадь под гиперболой}

% XXX
% $build$matter[print]: [[.], [.]]

\begingroup
    \def\abs#1{\lvert #1 \rvert}%

Будем обозначать $S[a; b]$ площадь под графиком $y = 1 / x$
на~отрезке $x \in [a; b]$, где $0 < a < b$.

Разобьем отрезок на~$n$ равных частей и~оценим $S[a; b]$ с~помощью
прямоугольничков:
\[
    \sum_{i=1}^{n}
        \frac{b - a}{n} \cdot \frac{1}{x_{i}}
=
    S_{n}[a; b]
\leq
    S[a; b]
\leq
    S^{n}[a; b]
=
    \sum_{i=1}^{n}
        \frac{b - a}{n} \cdot \frac{1}{x_{i-1}}
\; , \]
где $x_{i} = a + (b - a) \cdot i / n$.

Заметим, что разность между верхней и~нижней оценкой можно сделать сколь угодно
малой, изменяя $n$:
\[
    S^{n}[a; b] - S_{n}[a; b]
=
    \frac{b - a}{n} \cdot \left( \frac{1}{b} - \frac{1}{a} \right)
\]

Также из~формулы для $S_{n}[a; b]$ видно, что $S_{n}[ka; kb] = S_{n}[a; b]$
для любого положительного $k$.
Аналогично для $S^{n}[a; b]$.
Получаем, что
\[
    S_{n}[a; b] \leq S[ka; kb] \leq S^{n}[a; b]
\quad\text{и}\quad
    S_{n}[a; b] \leq S[a; b] \leq S^{n}[a; b]
\, , \]
откуда
\[
    \abs{S[ka; kb] - S[a; b]} \leq \abs{S^{n}[a; b] - S_{n}[a; b]}
\, . \]
Так как правую часть можно сделать сколь угодно малой, меняя $n$, то~левая
не~может быть положительным числом.
Отсюда $S[ka; kb] = S[a; b]$.

Геометрически это можно понять следующим образом: растянем график $y = 1/x$
в~$k$ раз от~оси $Oy$ и~сожмем в~$k$ раз к~оси $Ox$.
Тогда график перейдет в~себя, а~фигура, соответствующая $S[a; b]$, перейдет
в~$S[ka; kb]$.
Площадь при этом сначала увеличится в~$k$ раз, а~потом уменьшится.

Доопределим $S[b; a] = - S[a; b]$ при $b > a > 0$.
Тогда $S[a; b] + S[b; c] = S[a; c]$ при всех положительных $a$, $b$, $c$.

\subsection*{Логарифм}

Обозначим $\ln(x) = S[1; x]$, где $x > 0$.
Тогда
\[
    \ln(x y)
=
    S[1; x y]
=
    S[1; x] + S[x; x y]
=
    S[1; x] + S[1; y]
=
    \ln(x) + \ln(y)
\]

Отметим другие важные свойства логарифма:
\begin{itemize}
\item
$\ln(x)$ положителен при $x > 1$, и~отрицателен при $0 < x < 1$.
\item
$\ln(x)$ возрастает.
\item
$\ln(1/x) = - \ln(x)$.
\end{itemize}

\begin{problems}

\item
Докажите неравенства
\begin{problemeq*}
    (a - 1) / a < \ln(a) < a - 1
\end{problemeq*}
при $a > 0$.

\item
\subproblem
Докажите, что для каждого натурального $n$
\[
    \frac{1}{2} + \frac{1}{3} + \frac{1}{4} + \ldots + \frac{1}{n}
\leq
    \ln(n)
\leq
    \frac{1}{1} + \frac{1}{2} + \frac{1}{3} + \ldots + \frac{1}{n-1}
\; . \]
\subproblem
Докажите, что все части в~предыдущем неравенстве можно сделать сколь угодно
большими при изменении $n$.

\item
Пусть $n$~--- натуральное.
Сравните числа
\[
    \frac{1}{n+1} + \frac{1}{n+2} + \ldots + \frac{1}{2 n}
\quad\text{и}\quad
    \ln(2)
\]
и~докажите, что они отличаются не~более чем на~$1 / (2 n)$.

\item
Докажите неравенство
\begin{problemeq*}
    \ln(a) < (a^2 - 1) / (2 a)
\end{problemeq*}
при $a > 1$,
оценив площадь под графиком $y = 1 / x$ с~помощью трапеции.

\item
Докажите неравенство
\begin{problemeq*}
    2 (a - 1) / (a + 1) < \ln(a)
\end{problemeq*}
при $a > 1$,
проведя касательную к~гиперболе в~середине отрезка $[1; a]$.

\item
Докажите неравенства:
\\[0.3ex]
\subproblem $\ln(2) > 7 / 12$;
\qquad
\subproblem $\ln(2) > 2 / 3$;
\\[0.5ex]
\subproblem $\ln(2) < 5 / 6$;
\qquad
\subproblem $\ln(2) < 3 / 4$;
\qquad
\subproblem $\ln(2) < 17 / 24$.

\item
Докажите неравенство: $\ln 10 > 2$.

\item
Найдите целые части чисел
$\ln(4)$, $\ln(9)$, $\ln(8)$.

\item
Найдите целую часть числа
$\ln(25)$.

%\item
%Найдите целую часть чисел:
%\\[0.3ex]
%\subproblem $\ln(36)$;
%\quad
%\subproblem $\ln(25)$;
%\quad
%\subproblem $\ln(81)$;
%\quad
%\subproblem $\ln(100)$;
%\quad
%\subproblem $\ln(225)$.
%
%\item
%Найдите целую часть чисел:
%\\[0.3ex]
%\subproblem $\ln(8)$;
%\quad
%\subproblem $\ln(27)$;
%\quad
%\subproblem $\ln(1000)$.

\end{problems}

\endgroup % \def\abs

