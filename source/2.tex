% $date: 2016-06-29

\section*{Геометрические свойства парабол}

\begin{problems}

\itemy{0}
Пусть в~какой-то системе координат парабола задается уравнением
$y = a x^2 + b x + c$ ($a \neq 0$).
Докажите, что найдется система координат, где уравнение этой параболы будет
$y = x^2$.

\end{problems}

\definition
Парабола~--- это геометрическое место точек, равноудаленных
от~прямой~$d$ \emph{(директриса)}
и~точки~$F$ \emph{(фокус)}.
(Фокус не~лежит на~директрисе.)

\begin{problems}

\item
Докажите, что есть система координат, в~которой описанное ГМТ задается
уравнением $y = x^2$.
Как в~этой системе задаются фокус и~директриса?

\item
На~чистом листе бумаги нарисована парабола.
Циркулем и~линейкой постройте ее:
\\
\subproblem ось симметрии;
\qquad
\subproblem вершину.
% СР
%\\
%\subproblem фокус;
%\qquad
%\subproblem директрису.

%\end{problems}

%\definition
%\emph{Касательной} к~параболе называется прямая, имеющая ровно одну общую
%точку с~параболой и~не~параллельная оси параболы.

%\begin{problems}

\item
\subproblem
Из~точки~$A$ на~параболе опустили перпендикуляр~$AB$ на~директрису.
Докажите, что серединный перпендикуляр к~отрезку~$FB$ является касательной
к~параболе в~точке~$A$.
\\
\subproblem
Из~точки~$A$ на~параболе опустили перпендикуляр~$AC$ на~касательную
к~вершине $O$ параболы.
Докажите, что касательная к параболе в точке~$A$ делит отрезок $OC$ пополам.

\begin{figure}[h]
\begin{center}
    \jeolmfigure[width=0.25\linewidth]{optic}
\end{center}
\end{figure}

\item \emph{Оптическое свойство параболы}
Докажите, что все лучи, параллельные оси симметрии параболы, отразившись
от~параболы, пересекутся в фокусе.

\item
В~предыдущем листочке мы уже видели, что для каждой параболы геометрическое
место точек, из~которых она видна под прямым углом,~--- директриса.
Докажите теперь это свойство геометрическими методами (без вычислений).

\item
Пусть касательные к~параболе в~трех различных точках образуют
треугольник $PQR$.
Докажите, что фокус~$F$ параболы лежит на~описанной окружности этого треугольника.
% XXX картинка?

\end{problems}

