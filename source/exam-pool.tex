\section*{Задачи по алгебре для экзамена}

\begingroup
    \def\abs#1{\lvert #1 \rvert}%

Задачи, которые мы хотим дать в вариант:

\begin{problems}

% 3
\item\label{problem:area sample 3}%
Найдите площадь фигуры (рис.~\ref{fig:area sample 3}), ограниченной
параболой $y = x^2$ и~отрезками, соединяющими точки $(-1, 1)$ и~$(1, 1)$
с~точкой $(0, 2)$.

\begin{figure}[ht]\begin{center}
    \jeolmfigure[width=0.225\linewidth]{/3/area-sample-3}
    \caption{к~задаче~\ref{problem:area sample 3}.}
    \label{fig:area sample 3}
\end{center}\end{figure}

% 1
\item
Найдите все $a$, для каждого из~которых уравнение $\abs{x^2 - 5 x + 6} = a x$
имеет ровно три корня.

\end{problems}

Задачи, которые мы предлагаем взять, если останется место в варианте:

\begin{problems}

% 4
\item
Найдите сумму: $1^4 + 2^4 + 3^4 + \ldots + n^4$;

% 2
\item
На~параболе $y = x^2$ отмечены точки $A = (a, a^2)$ и~$B = (b, b^2)$.
Найдите между ними точку $M = (m, m^2)$, для которой сумма площадей двух
сегментов, ограниченных параболой и~отрезками $AM$ и~$BM$, наименьшая.

%% 5
%\item
%Найдите целую часть $\ln(49)$.

\end{problems}

\endgroup % \def\abs

