% $date: 2016-07-03

\section*{Суммирование степеней}

% XXX
% $build$matter[print]: [[.], [.]]

% $matter[-no-header]:
% - .[no-header]

\begingroup
    \def\abs#1{\lvert #1 \rvert}%
    \def\Abs#1{\left \lvert #1 \right \rvert}%
    \def\binom#1#2{\mathrm{C}_{#1}^{#2}}%
    \ifdefined\mathup
        \def\diff{\mathop{\mathup{\Delta}}}%
    \else
        \def\diff{\mathop{\Delta}}%
    \fi
    \def\sumpow#1#2{S^{#1}_{#2}}%

Мы хотим найти формулу для $\sumpow{k}{n} = 1^{k} + 2^{k} + \ldots n^{k}$.

\claim{Бином Ньютона}
Верно следующеее разложение
\[
    (a + b)^{k}
=
    a^{k} + \binom{k}{1} \cdot a^{k-1} b + \binom{k}{2} \cdot a^{k-2} b^{2}
    + \ldots +
    \binom{k}{k-1} \cdot a b^{k-1} + b^{k}
\, , \]
%где $\binom{k}{m} = k! / (m! (k - m)!$.
где $\binom{k}{m} = \dfrac{k!}{m! (k - m)!}$.

В~частности,
\[
    (n + 1)^{k+1}
=
    n^{k+1} + (k + 1) \cdot n^{k} + \binom{k+1}{2} \cdot n^{k-1}
    + \ldots +
    \binom{k+1}{k} \cdot n + 1
\, , \]
или, в~терминах разностей,
\[
    \diff(n^{k+1})
=
    (k + 1) \cdot n^{k} + \binom{k+1}{2} \cdot n^{k-1}
    + \ldots +
    \binom{k+1}{k} \cdot n + 1
\, . \]

С~другой стороны,
\begin{gather*}
    \diff(\sumpow{1}{n-1}) = n^{1}
\, , \quad
    \diff(\sumpow{2}{n-1}) = n^{2}
\, , \quad \ldots, \quad
    \diff(\sumpow{k}{n-1}) = n^{k}
\, . \end{gather*}
Мы хотим выразить $\sumpow{k}{n}$, то~есть нужно подобрать последовательность,
$\diff$ которой равна $n^{k}$.
Воспользуемся тем, что $n^{k}$ фигурирует в~$\diff(n^{k+1})$, а~остальные
слагаемые там выражаются через $\diff(\sumpow{l}{n-1})$, $l < k$.

Рассмотрим последовательность
\[
    x_{n}
=
    n^{k+1} - \binom{k+1}{2} \cdot \sumpow{k-1}{n-1}
    - \ldots
    - \binom{k+1}{k} \cdot \sumpow{1}{n-1}
    - n
\, . \]
Тогда $\diff(x_{n}) = (k + 1) \cdot n^{k}$.
Это еще не~означает, что $x_{n} / (k + 1)$ совпадает с~$\sumpow{k}{n-1}$,
так как последовательности с~одинаковой $\diff$ могут различаться на~константу.

Заметим, что $\sumpow{k}{0} = 0$ и, внезапно, $x_{1} = 0$.
Значит, последовательности действительно совпадают.
Поправив индексы, получаем
\[
    \sumpow{k}{n}
=
    \frac{(n + 1)^{k+1}}{k + 1}
    -
    \frac{
        \binom{k+1}{2} \cdot \sumpow{k-1}{n}
        + \ldots +
        \binom{k+1}{k} \cdot \sumpow{1}{n}
        + n + 1
    }{k + 1}
\; . \]
Отметим, что это многочлен степени $k + 1$ (доказывается по индукции),
и коэффициент при старшей степени его равен $1 / (k + 1)$.


\subsection*{Площадь под графиком многочлена}

Мы хотим найти площадь $I_{a}$ под графиком $y = x^{k}$ на~отрезке $[0; a]$.
Разобъем отрезок на~$n$ частей, и~оценим площадь с~помощью прямоугольничков:
\[
    \frac{a}{n}
    \cdot
    \sum_{m=0}^{n-1}
        (m a / n)^{k}
\leq
    I_{a}
\leq
    \frac{a}{n}
    \cdot
    \sum_{m=1}^{n}
        (m a / n)^{k}
\, . \]
Вынеся $(a / n)^{k}$ за~знаки суммирования, получаем
\[
    \frac{a^{k+1}}{n^{k+1}}
    \cdot
    \sum_{m=0}^{n-1}
        m^{k}
\leq
    I_{a}
\leq
    \frac{a^{k+1}}{n^{k+1}}
    \cdot
    \sum_{m=1}^{n}
        m^{k}
\, . \]
Следовательно,
\[
    \frac{a^{k+1}}{n^{k+1}}
    \sumpow{k}{n-1}
\leq
    I_{a}
\leq
    \frac{a^{k+1}}{n^{k+1}}
    \sumpow{k}{n}
\, . \]
Заменим
\[
    \sumpow{k}{n-1} = n^{k+1} / (k + 1) + P_{k}(n)
\, , \quad
    \sumpow{k}{n} = n^{k+1} / (k + 1) + Q_{k}(n)
\, ,\]
где $P_{k}$~--- многочлен степени не~более $k$, как мы установили ранее,
а~$Q_{k}(n) = P_{k}(n) + \bigl( (n + 1)^{k+1} - n^{k+1} \bigr) / (k + 1)$~---
также многочлен степени не~более $k$.
Тогда
\[
    \Abs{I_{a} - \frac{a^{k+1}}{k + 1}}
\leq
    \max\left(
        \frac{\abs{Q_{k}(n)}}{n^{k+1}}
    ,
        \frac{\abs{P_{k}(n)}}{n^{k+1}}
    \right)
\, . \]
Дробь справа может быть сколь угодно близкой к~нулю при достаточно больших $n$,
так как это отношение многочленов разных степеней.
Значит, число слева равно нулю, то~есть
\[
    I_{a} = \frac{a^{k+1}}{k + 1}
\ . \]

\endgroup % \def\abs \def\binom \def\diff \def\sumpow

