% $date: 2016-06-30

\section*{Площади, связанные с параболой}

% XXX
% $build$matter[print]: [[.], [.]]

% $matter[-preamble-package-guard]:
% - preamble package: subcaption
% - .[preamble-package-guard]

\setcounter{figure}{0}

\begin{problems}

\item
Площадь треугольника с~вершинами
$A = (x_{A}, y_{A})$, $B = (x_{B}, y_{B})$, $C = (x_{C}, y_{C})$
задается формулой
\[
    S_{ABC}
=
    \frac{1}{2}
    \bigl \lvert
          x_{A} \cdot y_{B} - x_{B} \cdot y_{A}
        + x_{B} \cdot y_{C} - x_{C} \cdot y_{B}
        + x_{C} \cdot y_{A} - x_{A} \cdot y_{C}
    \bigr \rvert
.\]
Докажите эту формулу по~крайней мере при одном нетривиальном взаимном
расположении точек.

\end{problems}

\begin{minipage}{0.795\linewidth} \begin{problems}
\item
\subproblem
Разбив отрезок $[0; a]$ на~$n$~частей, докажите, что площадь $I_{0;a}$ фигуры
под параболой $y = x^2$ на~этом отрезке ограничена неравенствами
\[
    \frac{a^3}{n^3} \cdot
    \frac{n (n - 1) (2 n - 1)}{6}
\leq
    I_{0;a}
\leq
    \frac{a^3}{n^3} \cdot
    \frac{n (n + 1) (2 n + 1)}{6}
\;.\]
\subproblem
Докажите, что $I_{0;a} = a^3 / 3$.
\end{problems} \end{minipage}
\hfill
\begin{minipage}{0.195\linewidth}
    \jeolmfigure[width=\linewidth]{area-under-parabola-2}
\end{minipage}

\begin{problems}

\item\label{problem:quadratic integral}%
Пусть $I_{a;b}$~--- площадь криволинейной трапеции под параболой $y = x^2$
на~отрезке $[a; b]$.
Докажите, что
\[
    I_{a;b}
=
    \frac{b^3 - a^3}{3}
\;.\]

\begin{figure}[ht]\begin{center}
\strut\hfill
    \begin{subfigure}{0.45\textwidth}\begin{center}
        \jeolmfigure[width=0.5\linewidth]{archimedes}
        \caption{к~лемме Архимеда
            (задача~\ref{problem:parabolic segment area}).}
        \label{fig:parabolic segment area}
    \end{center}\end{subfigure}
\hfill
    \begin{subfigure}{0.45\textwidth}\begin{center}
        \jeolmfigure[width=0.5\linewidth]{archimedes-midpoints}
        \caption{к~задаче~\ref{problem:parabolic triangle midpoints}.}
        \label{fig:parabolic triangle midpoints}
    \end{center}\end{subfigure}
\hfill\strut
\caption{}
\end{center}\end{figure}

\item\label{problem:parabolic segment area}%
Пусть $A = (a, a^2)$ и~$B = (b, b^2)$~--- точки параболы $y = x^2$,
и~касательные к~ней в~этих точках пересекаются в~точке~$C$
(рис.~\ref{fig:parabolic segment area}).
\\
\subproblem
Найдите площадь сегмента $AB$ параболы, представив её~как разность трапеции
$(a,0)$, $A$, $B$, $(b,0)$ и~криволинейной трапеции из~задачи
\ref{problem:quadratic integral};
\\
\subproblem
Найдите $S_{ABC}$ и~докажите \emph{лемму Архимеда:}
\[
    S_{\text{сегмент $AB$}}
=
    \frac{2}{3} \cdot S_{ABC}
\]

\item\label{problem:area sample 1}%
Найдите площадь фигуры (рис.~\ref{fig:area sample 1}), ограниченной графиками
функций $f(x) = x^2$ и~$g(x) = 2 x - x^2$.

\begin{figure}[ht]\begin{center}
%\strut\hfill
%    \begin{subfigure}{0.45\textwidth}\begin{center}
        \jeolmfigure[width=0.225\linewidth]{area-sample-1}
        \caption{к~задаче~\ref{problem:area sample 1}.}
        \label{fig:area sample 1}
%    \end{center}\end{subfigure}
%\hfill
%    \begin{subfigure}{0.45\textwidth}\begin{center}
%        \jeolmfigure[width=0.5\linewidth]{area-sample-2}
%        \caption{к~задаче~\ref{problem:area sample 2}.}
%        \label{fig:area sample 2}
%    \end{center}\end{subfigure}
%\hfill\strut
%\caption{}
\end{center}\end{figure}

% СР
%\item\label{problem:area sample 2}%
%Найдите площадь фигуры (рис.~\ref{fig:area sample 2}), ограниченной
%параболой $y = 1 - x^2$ и~отрезками, соединающими точки $(-1, 0)$ и~$(1, 0)$
%с~точкой $(0, 2)$.

\item\label{problem:parabolic triangle midpoints}%
Пусть $A$ и~$B$~--- точки параболы, и~касательные к~параболе в~этих точках
пересекаются в~точке~$C$
(рис.~\ref{fig:parabolic triangle midpoints}).
\\
\subproblem
Пусть $M$~--- середина~$AB$.
Докажите, что прямая~$CM$ параллельна оси параболы (т.~е. точка~$C$ лежит точно
<<под>> точкой~$M$).
\\
\subproblem
Пусть $T$~--- точка пересечения $CM$ с~параболой.
Докажите, что $T$ делит отрезок~$CM$ пополам.
\\
\subproblem
Докажите, что касательная к~параболе в~точке~$T$ параллельна прямой~$AB$
и~содержит среднюю линию треугольника $ABC$.

\item
На~параболе $y = p x^2 + q x + r$ ($p \neq 0$) даны две точки $A$ и~$B$
с~абсциссами $a$ и~$b$ соответственно.
Найдите $d$ такое, что прямая $x = d$ делит сегмент~$AB$ параболы пополам
(по~площади).

\item
Пусть $A$, $B$, $C$~--- точки параболы $y = x^2$, а~касательные к~ней
в~этих точках попарно пересекаются в~точках $P$, $Q$, $R$.
Докажите, что $S_{ABC} = 2 S_{PQR}$.

% КР
%\item
%На~параболе $y = x^2$ отмечены точки $A = (a, a^2)$ и~$B = (b, b^2)$.
%Найдите между ними точку $M = (m, m^2)$, для которой сумма площадей двух
%сегментов, ограниченных параболой и~отрезками $AM$ и~$BM$ наименьшая.

\end{problems}

